% bibliografia em outras línguas: http://en.wikibooks.org/wiki/LaTeX/Bibliography_Management

\documentclass[12pt,a4paper,utf8]{ppgsi}

\usepackage{url}

\title{Arquitetura ARMv8-A}

\coverauthor{Bruno Lourenço da Cunha\\Kevin Gabriel Gonçalves de Oliveira\\ 
Renan Ernesto Silva Pinto\\Vinicius Alves Matias}

\author{Bruno Lourenço da Cunha\inst{1}
    \and Kevin Gabriel Gonçalves de Oliveira\inst{2}
    \and \\Renan Ernesto Silva Pinto\inst{3}
    \and Vinicius Alves Matias\inst{4}}

\address{Escola de Artes, Ciências e Humanidades -- Universidade de São Paulo\\
  São Paulo -- SP, Brazil
  \email{bruno\_cunha@usp.br}
  \nextinstitute
    Escola de Artes, Ciências e Humanidades -- Universidade de São Paulo\\
    São Paulo -- SP, Brazil
    \email{autor2@email.com}
    \nextinstitute
    Escola de Artes, Ciências e Humanidades -- Universidade de São Paulo\\
    São Paulo -- SP, Brazil
    \email{renan.ernesto@usp.br}
    \nextinstitute
    Escola de Artes, Ciências e Humanidades -- Universidade de São Paulo\\
    São Paulo -- SP, Brazil
    \email{autor4@email.com}
}

\numero{000/2019}
\mes{11}
\ano{2019}

\begin{document}
\maketitle

\begin{abstract} 
    Relatório técnico referente à arquitetura ARMv8-A, desenvolvida pela ARM Holdings e presente em diversos chips fabricados por empresas como Apple Inc., Nvidia e a própria ARM Holdings.
\end{abstract}

\section{Histórico}
    A arquitetura ARMv8-A foi anunciada em outubro de 2011 pela empresa britânica ARM (Advanced RISC Machine) Holdings como a oitava versão da arquitetura ARM, tendo seu perfil definido para aplicações e portanto, sendo otimizada para sistemas operacionais de alto nível. 
    A arquitetura pertencente à linha de arquitetura RISC (Reduced Instruction Set Computer) contendo as seguintes características:
    \begin{itemize}
      \item Um grande arquivo uniforme de registro.
      \item Uma arquitetura de \textit{load/store}, onde as operações de processamento de dados operam somente no conteúdo do registrador e não diretamente no conteúdo da memória.
      \item Modo simples de endereçamento, com todos os endereços de \textit{load/store} determinados somente do conteúdo do registrador e campos da intrução.
    \end{itemize}
    A arquitetura suporta tanto endereçamento quanto aritmética de 64 bits e instruções de tamanho fixo de 32 bits, além de um estado de execução de 64 bits (AArch64) e outro de 32 bits (AArch32), que é completamente compatível com as versões anteriores da arquitetura ARM.
    Atualmente, a arquitetura vem sendo aprimorada e está em constante evolução. A ARMv8.6-A fornece um ambiente propício para o desenvolvimento de Redes Neurais (NN) para Machine Learning (ML) através de General Matrix Multiply (GEMM) e BFloat 16. Além de todas evoluções presentes nas versões ARMv8.1-A, ARMv8.2-A, ARMv8.3-A, ARMv8.4-A e ARMv8.5-A.
    
   
 \section{Uso atual}
 A arquitetura está presente em diversos chips que visam uma boa eficiência energética aliada a um alto desempenho. Um exemplo de chip com tais características é o Snapdragon 855, presente em celulares como Asus Zenfone 5Z\footnote{\url{https://www.asus.com/Phone/ZenFone-5Z-ZS620KL/Tech-Specs/} .}, Xiaomi Mi 8\footnote{\url{https://www.mi.com/global/mi8/specs} .} e vários outros\footnote{\url{https://www.techwalls.com/qualcomm-snapdragon-855-smartphones/} .} que usam as microarquiteturas Cortex-A76 e Cortex-A55.
 
\section{Desempenho}
Para medir o desempenho da arquitetura, o dispositivo usado como referência foi o OnePlus 7 Pro, equipado com o chip Snapdragon 855 (anunciado em 5 de Dezembro de 2018 pela Qualcomm Technologies, baseado na arquitetura ARMv8-A). Os testes foram realizados a partir da versão 5.0.3 do Geekbench para Android AArch64. Como referência, o Geekbench estabelece 1000 pontos como sendo o resultado da pontuação de um i3-8100. Os resultados foram os seguintes:
\begin{itemize}
      \item Single-Core Score: 763 Pontos
      \item Single-Core Crypto Score: 1027 Pontos
      \item Single-Core Integer Score: 734 Pontos
      \item Single-Core Floating Point Score: 781 Pontos
      \item Multi-Core Score: 2778 Pontos
      \item Multi-Core Crypto Score: 3974 Pontos
      \item Multi-Core Integer Score: 2707	Pontos
      \item Multi-Core Floating Point Score: 2731 Pontos
      
\end{itemize}

\section{Instruction Set}
\subsection{Estrutura básica}
O assembler da arquitetura ARMv8-A reconhece instruções tanto em caixa alta como em caixa baixa. As instruções são linhas compostas por um ou mais rótulos(labels) seguidos do código da instrução, conhecido como OPCODE, um registrador de destino e um ou mais registradores, separados por vírgula, utilizados na operação. Sendo assim, a estrutura das instruções segue o seguinte padrão:
\\\centerline{\{label:*\} \{opcode \{dest\{, source1\{, source2\{, source3\}\}\}\}\}}
\\A ordem do registrador de destino e dos registradores fonte são trocadas apenas na instrução store.
\begin{itemize}
  \item item
\end{itemize}
Teste2.


\bibliographystyle{ppgsi}
\bibliography{bibitex}
\nocite{manual}
\nocite{pressreal}
\nocite{pressarmv86}
\nocite{pressSdragon}
\nocite{bench}

\end{document}
