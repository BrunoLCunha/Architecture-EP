% bibliografia em outras línguas: http://en.wikibooks.org/wiki/LaTeX/Bibliography_Management

\documentclass[12pt,a4paper,utf8]{ppgsi}

\usepackage{url}

\title{Arquitetura ARMv8-A}

\coverauthor{Bruno Lourenço da Cunha\\Kevin Gabriel Gonçalves de Oliveira\\ 
Renan Ernesto Silva Pinto\\Vinicius Alves Matias}

\author{Bruno Lourenço da Cunha\inst{1}
    \and Kevin Gabriel Gonçalves de Oliveira\inst{2}
    \and \\Renan Ernesto Silva Pinto\inst{3}
    \and Vinicius Alves Matias\inst{4}}

\address{Escola de Artes, Ciências e Humanidades -- Universidade de São Paulo\\
  São Paulo -- SP, Brazil
  \email{bruno\_cunha@usp.br}
  \nextinstitute
    Escola de Artes, Ciências e Humanidades -- Universidade de São Paulo\\
    São Paulo -- SP, Brazil
    \email{autor2@email.com}
    \nextinstitute
    Escola de Artes, Ciências e Humanidades -- Universidade de São Paulo\\
    São Paulo -- SP, Brazil
    \email{autor3@email.com}
    \nextinstitute
    Escola de Artes, Ciências e Humanidades -- Universidade de São Paulo\\
    São Paulo -- SP, Brazil
    \email{autor4@email.com}
}

\numero{000/2019}
\mes{09}
\ano{2019}

\begin{document}
\maketitle

\begin{abstract} 
    Relatório técnico referente à arquitetura ARMv8-A, desenvolvida pela ARM Holdings e presente em diversos chips fabricados por empresas como Apple Inc., Nvidia e a própria ARM Holdings.
\end{abstract}

\section{Histórico}
    A arquitetura ARMv8-A foi anunciada em Outubro de 2011 pela empresa britânica ARM (Advanced RISC Machine) Holdings como a oitava versão da arquitetura ARM, tendo seu perfil definido para aplicações e portanto, sendo otimizada para sistemas operacionais de alto nível. 
    A arquitetura pertencente à linha de arquitetura RISC (Reduced Instruction Set Computer) contendo as seguintes características:
    \begin{itemize}
      \item Um grande arquivo uniforme de registro.
      \item Uma arquitetura de \textit{load/store}, onde as operações de processamento de dados operam somente no conteúdo do registrador e não diretamente no conteúdo da memória.
      \item Modo simples de endereçamento, com todos os endereços de \textit{load/store} determinados do conteúdo do registrador e campos da intrução somente.
    \end{itemize}
    A arquitetura suporta tanto endereçamento quanto aritmética de 64 bits e instruções de tamanho fixo de 32 bits, além de um estado de execução de 64 bits (AArch64) e outro de 32 bits (AArch32), que é completamente compatível com as versões anteriores da arquitetura ARM. 
    Atualmente, a arquitetura está presente em diversos chips que visam o mínimo consumo de energia. Um exemplo é o Snapdragon 845 (presente em celulares como Asus Zenfone 5Z\footnote{\url{https://www.asus.com/Phone/ZenFone-5Z-ZS620KL/Tech-Specs/} .}, Xiaomi Mi 8\footnote{\url{https://www.mi.com/global/mi8/specs} .} e vários outros\footnote{\url{https://www.techwalls.com/qualcomm-snapdragon-845-smartphones/} .}), cuja microarquitetura é a Cortex-A75.
    
\bibliographystyle{ppgsi}
\bibliography{bibitex}
\nocite{manual}

\end{document}
